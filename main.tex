% !TeX root = main.tex
%!TEX program = lualatex
\documentclass[ paper=a4paper, 12pt]{article}

\usepackage{./preamble}
% === Custom Commands (Course-Specific) ===
\newcommand{\courseTitle}{EE0000 某科目} % Course title
\newcommand{\reportTitle}{實驗結果報告} % Report title
\newcommand{\experimentTitle}{實驗主題} % Experiment title
\newcommand{\professors}{某教授} % Professors' names
\newcommand{\department}{某系} % Department
\newcommand{\studentID}{114XXXXXX} % Student ID
\newcommand{\studentName}{王小明} % Student name
\newcommand{\groupNumber}{第 0 組} % Group number
\newcommand{\dB}{\mathrm{dB}} % Decibel unit
\newcommand{\dBW}{\mathrm{dBW}} % Decibel-watt unit
\newcommand{\dBm}{\mathrm{dBm}} % Decibel-milliwatt unit


\begin{document}


\begin{titlepage}
    \begin{center}
        
    
        \vspace*{2cm}
        
        {\Huge \bfseries \underline{\courseTitle} \par}
        {\Huge \bfseries \underline{\reportTitle} \par}
        \vspace{1cm}
        { \Huge \experimentTitle \par}
        
        \vspace{1cm}
        {\Huge 指導教授:\professors \par}
        
        \vspace{3cm}
        {\Huge
        \begin{center}
            系級:\department \\
            學號:\studentID \\
            姓名:\studentName \\
            組別:\groupNumber
        \end{center}
        }
    
        \vfill
    \end{center}
    \end{titlepage}


\section{引言}
本次實驗的目的是使學生熟悉向量網路分析儀(VNA)的基本操作,並測量待測元件(DUT)的 S 參數。實驗使用 R\&S ZVL 網路分析儀與 LiteVNA 來表徵微波元件,包括帶通濾波器、衰減器和傳輸線。其他目標包括理解向量與純量網路分析儀的差異、執行校準,以及計算相關物理量,如反射係數(\(\Gamma\))、反射損耗(RL)、電壓駐波比(VSWR)和插入損耗(IL)。
\section{Test Chapter}
		\begin{Thm}{Test Theorem}{}
			This is the content of The Test Theorem.
		\end{Thm}
		\begin{Lem}{Test Lemma}{}
			This is the content of The Test Lemma.
		\end{Lem}
		\begin{Cor}{Test Corollary}{}
			This is the content of The Test Corollary.
		\end{Cor}
		\begin{Def}{Test Definition}{}
			This is the content of The Test Definition.
		\end{Def}
		\begin{Exp}{The Test Example}{}
		 This is the content of The Test Example.
		\end{Exp}
\section{理論}
\subsection{S 參數}
S 參數描述線性網路的電氣特性,通過其端口的入射波與反射波關係來定義。對於雙端口網路,S 矩陣定義為:

\[
\begin{bmatrix}
b_1 \\
b_2
\end{bmatrix}
=
\begin{bmatrix}
S_{11} & S_{12} \\
S_{21} & S_{22}
\end{bmatrix}
\begin{bmatrix}
a_1 \\
a_2
\end{bmatrix}
\]

其中:
\begin{itemize}
\item \(a_1, a_2\):端口 1 和端口 2 的入射波電壓,
\item \(b_1, b_2\):端口 1 和端口 2 的反射波電壓。

\end{itemize}

各 S 參數的定義如下:
\begin{itemize}
\item \(S_{11} = \left. \frac{b_1}{a_1} \right|_{a_2=0}\):當端口 2 匹配時,端口 1 的反射係數,
\item \(S_{21} = \left. \frac{b_2}{a_1} \right|_{a_2=0}\):當端口 2 匹配時,從端口 1 到端口 2 的傳輸係數,
\item \(S_{12} = \left. \frac{b_1}{a_2} \right|_{a_1=0}\):當端口 1 匹配時,從端口 2 到端口 1 的傳輸係數,
\item \(S_{22} = \left. \frac{b_2}{a_2} \right|_{a_1=0}\):當端口 1 匹配時,端口 2 的反射係數。

\end{itemize}

\subsection{S 參數的性質}
\begin{itemize}
\item \textbf{匹配端口}:對於完全匹配的端口 \(i\),\(S_{ii} = 0\)。
\item \textbf{互易性}:對於互易網路,\(S_{nm} = S_{mn}\)。
\item \textbf{被動性}:對於被動網路,\(|S_{ij}| \leq 1\)。
\item \textbf{無損網路}:對於無損且互易的網路,對第 \(i\) 端口,\(\sum_{n=1}^N |S_{ni}|^2 = 1\)。

\end{itemize}

\subsection{相關物理量}
\begin{itemize}
\item \textbf{反射損耗(Reflextion Loss,RL)}:
\[
\text{RL} = -20 \log |\Gamma| = -20 \log |S_{11}| \text{ (dB)}
\]
反射損耗越高,表示反射越小。
\item \textbf{插入損耗(Insertion Loss,IL)}:
\[
\text{IL} = -20 \log |T| = -20 \log |S_{21}| \text{ (dB)}
\]
插入損耗越低,表示傳輸損失越小。
\item \textbf{電壓駐波比(VSWR)}:
\[
\text{VSWR} = \frac{1 + |\Gamma|}{1 - |\Gamma|}
\]
其中 \(\Gamma = S_{11}\),適用於端口 2 匹配的情況。
\end{itemize}


註:第二份文件中將 RL 定義為 \(20 \log \left| \frac{1}{S_{11}} \right|\),IL 定義為 \(20 \log \left| \frac{1}{S_{21}} \right|\),這與入射波與反射/傳輸波的比例一致。然而,微波工程的標準慣例(例如 Pozar, 1998)使用上述定義,本報告採用此標準以保持一致性。

\section{實驗設備}
使用的設備包括:
\begin{itemize}
\item R\&S ZVL 向量網路分析儀(頻率範圍 9 kHz–13.6 GHz,動態範圍 110 dB),
\item ZV-Z136 校準套件,
\item 帶通濾波器,
\item 衰減器,
\item 傳輸線,
\item N-SMA 轉接頭。
\end{itemize}


網路分析儀配置了掃頻合成源,通過 PORT 1 和 PORT 2 連接到待測元件,並進行校準以確保測量精度。

\section{實驗步驟}
\subsection{校準}
1. 通過「S 參數嚮導」軟鍵選擇 ZV-Z136 校準套件。
2. 設置頻率範圍(9 kHz–13.6 GHz)和輸出功率(-10 dBm)。
3. 在 PORT 1 和 PORT 2 分別連接並測量開路(open)、短路(short)和匹配(match)標準件。
4. 將 PORT 1 和 PORT 2 通過直通(through)標準件連接並測量。
5. 按下「DONE」完成完整的雙端口校準。

\subsection{測量}
1. 將待測元件(例如帶通濾波器、衰減器或傳輸線)連接到 PORT 1 和 PORT 2。
2. 使用「Scale」\(\rightarrow\)「AUTO SCALE」調整顯示。
3. 測量特定參數:
    \begin{itemize}
        \item \textbf{反射係數線性大小}:設置格式為「LIN MAG」。
   \item \textbf{VSWR}:設置格式為「SWR」。
   \item \textbf{Smith 圖}:設置格式為「SMITH CHART」以顯示 \(S_{11}\) 和 \(S_{22}\)。
    \end{itemize}

4. 使用標記(Marker)從顯示屏上讀取數值。

\section{實驗結果}
由於缺乏實際測量數據,本節以預期行為進行通用描述:
\begin{itemize}
\item \textbf{帶通濾波器}:在通帶內 \(|S_{21}|\) 應較高(低 IL),通帶外較低(高 IL);\(|S_{11}|\) 在通帶內應較低(良好匹配)。
\item \textbf{衰減器}:\(|S_{21}|\) 應顯示恆定衰減(例如理想情況下 13 dB),若匹配良好,\(|S_{11}|\) 和 \(|S_{22}|\) 接近零。
\item \textbf{傳輸線}:\(|S_{21}|\) 隨頻率增加因損耗而降低,若匹配良好,\(|S_{11}|\) 較低。
\begin{table}[ht]
\centering
\begin{tabular}{|c|l|l|l|c|c|}
\hline
\multirow{2}{*}{駐波標示} & \multirow{2}{*}{頻率 (GHz)} & \multirow{2}{*}{損耗 (dB)} & \multirow{2}{*}{駐波參考點} & \multicolumn{2}{c|}{Smith Chart} \\ \cline{5-6}
& & & & S11 & S22 \\ \hline
最大損耗 (mark1) & & & & & \\ \hline
左側3dB點 (mark2) & & & & & \\ \hline
右側3dB點 (mark3) & & & & & \\ \hline
操作損耗 (mark2-3) & & & & & \\ \hline
\multicolumn{4}{|c|}{} & S11 & S21 \\ \hline
\multicolumn{4}{|c|}{S 參數參考點} & S12 & S22 \\ \hline
\end{tabular}
\caption{帶通濾波器 (1455-1925MHz) 測量數據}
\end{table}

\subsection*{2. 10 dB 衰減器}
\begin{table}[h!]
\centering
\begin{tabular}{|c|l|l|l|c|c|}
\hline
\multirow{2}{*}{駐波標示} & \multirow{2}{*}{頻率 (GHz)} & \multirow{2}{*}{損耗 (dB)} & \multirow{2}{*}{駐波參考點} & \multicolumn{2}{c|}{Smith Chart} \\ \cline{5-6}
& & & & S11 & S22 \\ \hline
衰減效果 & 10 & & & & \\ \hline
\multicolumn{4}{|c|}{} & S11 & S21 \\ \hline
\multicolumn{4}{|c|}{S 參數參考點} & S12 & S22 \\ \hline
\end{tabular}
\caption{10 dB 衰減器測量數據}
\end{table}
\end{itemize}

完整的報告應包含 \(|S_{11}|\) 和 \(|S_{21}|\) 隨頻率變化的圖表以及 Smith 圖表示。

\section{討論}
網路分析儀測量有效表徵了待測元件的性能。校準消除了系統誤差(例如電纜損耗、失配),確保測量準確性。與純量網路分析儀不同,向量網路分析儀測量幅度和相位,能通過 Smith 圖分析阻抗。潛在誤差包括校準不當、連接器問題或因直流信號(依注意事項禁止)損壞待測元件。R\&S ZVL 的寬頻範圍(9 kHz–13.6 GHz)和高動態範圍(110 dB)提升了測量精度。

\section{結論}
本次實驗提供了 R\&S ZVL 和 LiteVNA 的實操經驗,增強了對 S 參數、其測量方法及校準重要性的理解。這些技能對於射頻和微波電路設計至關重要。

\section{作業問題}
\subsection{1. LiteVNA 工作原理與信號流程}
LiteVNA 是一款向量網路分析儀,通過內部掃頻合成器產生射頻信號,從 PORT 1 輸出。信號與待測元件交互,產生反射波(\(b_1\))和傳輸波(\(b_2\)),分別由 PORT 1 和 PORT 2 通過方向耦合器測量。接收器檢測幅度和相位,處理器根據比例(例如 \(S_{11} = b_1/a_1\))計算 S 參數。使用已知標準件校準以修正誤差,結果顯示為幅度圖或 Smith 圖。

\subsection{2. 向量與純量網路分析儀的差異}
\begin{itemize}
\item \textbf{向量}:測量幅度和相位,能全面表徵網路(例如阻抗、導納)。應用:射頻設計、Smith 圖分析。
\item \textbf{純量}:僅測量幅度(例如 \(|S_{11}|\)、\(|S_{21}|\))。應用:增益/損耗測量、較簡單系統。

\end{itemize}

\subsection{3. 反射係數轉換公式}
\[
\text{RL} = -20 \log |\Gamma| \text{ (dB)}
\]
\[
\text{VSWR} = \frac{1 + |\Gamma|}{1 - |\Gamma|}
\]

\subsection{4. 傳輸係數與插入損耗}
\[
\text{IL} = -20 \log |T| \text{ (dB)}
\]
其中 \(T = S_{21}\)。註:衰減常數 \(\alpha\) 適用於傳輸線(\(T = e^{-\alpha l}\)),但對於雙端口元件,IL 直接與 \(S_{21}\) 相關。

\subsection{5. 13 dB 理想衰減器的 S 參數}
對於理想、匹配、互易的衰減器,衰減 13 dB(\(S_{21} = 10^{-13/20} \approx 0.2239\)):
\[
S = \begin{bmatrix}
0 & 0.2239 \\
0.2239 & 0
\end{bmatrix}
\]

\subsection{6. VSWR = 1.5 時的反射功率}
給定 VSWR = 1.5:
\[
\frac{1 + |\Gamma|}{1 - |\Gamma|} = 1.5 \implies |\Gamma| = 0.2
\]
反射功率 = \(|\Gamma|^2 = 0.04\),即入射功率的 4\%。

\newpage
\section{問題討論}

\begin{enumerate}
\item 說明LiteVNA工作原理,並利用方塊圖解釋射頻信號由網路
分析儀輸出後,經過待測元件,再回到網路分析儀之流程及網
路分析儀內部如何處理訊號將結果顯示在螢幕上之流程(參考
網頁: https://nanovna.com/)。
\item 說明向量網路分析儀及純量網路分析儀之差異及應用範圍。
\item 寫出反射係數與反射損耗(return loss, RL)、電壓駐波比
(VSWR)間的轉換公式。
\item 寫出穿透係數T與衰減常數a、穿透損耗(insertion loss, IL)間
的轉換公式。
\item 寫出13 dB理想衰減器之二埠S參數。
\item 使用網路分析儀量測分析一個負載之電壓駐波比為1.5,試計
算出有多少功率會反射回網路分析儀。
\end{enumerate}




\end{document}

